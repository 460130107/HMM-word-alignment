\documentclass{article}
\usepackage{fullpage}
\usepackage{graphicx}
\usepackage{amsmath}

\begin{document}

\quote{
We compare the AER performance of IBM Models 1 and 2 using a couple of initialization methods with an HMM alignment model incorporating some heuristics for speed advantage. We demonstrate the effects of various symmetrization methods combining English-French and French-English alignments generated by all of these models, and discuss the efficacy of AER as a useful and informative metric and the alignment task in general.
}

\section{Initializing IBM Models}

Given sets of English and French sentences $E_1^M$ and $F_1^N$, for every French token $f$, we designate an initial set $S_f$ of possible candidate translations as

\[
    S_f = \bigcup_{i=1}^N E_i \text{if} f \in F_i
\]

Under the naive scheme, probability is distributed among these candidates uniformly:

\[
    p(f|e) = \frac{1}{|S_f|}
\]

A more nuanced approach obtains initial weights by exploiting the string-level commonalities between English and French words that arise due to their languages' linguistic proximity. Where ld is the Levenshtein distance function, we take the {\em edit ratio} between to strings $a$ and $b$ to be 
\[
    \text{er}(a,b) = \frac{\text{ld}(a, b)}{\max(|a|, |b|)} + \delta
\]
and redefine the initial translation probabilities as
\[
    p(f|e) = \frac{\text{er}(f,e)}{\sum_{i=1}^{|S_f|} \text{er}(f, {S_f}_i)}
\]
The smoothing term $\delta$ TODO. Results are shown in Table \ref{tbl:ed_dist}.
 %todo more explain

    \begin{table}[h]
\begin{center}
\begin{tabular}{l|ll}
    & \multicolumn{2}{c} {AER}\\
    initialization & Model 1 & Model 2\\ \hline
    uniform & & \\
    edit distance & &
\end{tabular}
\end{center}
\label{tbl:ed_dist}
\caption{Using a function of Levenshtein edit distance to seed Model 1 and Model 2 improves AER over the baseline uniform initialization.}
\end{table}

\section{An HMM alignment model}

thresholds and ranges for pruning and speedup

learning curve for iterations 1-7

\section{Symmetrization of forward and reverse alignments}

\section{AER and the Hansards alignments}

\section{Summary of results and analysis}

ablation table



\end{document}
